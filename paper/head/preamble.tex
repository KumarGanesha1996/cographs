% !TEX root = ../main.tex

%
% Niniejszy plik stanowi przykład formatowania pracy magisterskiej na
% Wydziale MIM UW.  Szkielet użytych poleceń można wykorzystywać do
% woli, np. formatujac wlasna prace.
%
% Zawartosc merytoryczna stanowi oryginalnosiagniecie
% naukowosciowe Marcina Wolinskiego.  Wszelkie prawa zastrzeżone.
%
% Copyright (c) 2001 by Marcin Woliński <M.Wolinski@gust.org.pl>
% Poprawki spowodowane zmianami przepisów - Marcin Szczuka, 1.10.2004
% Poprawki spowodowane zmianami przepisow i ujednolicenie 
% - Seweryn Karłowicz, 05.05.2006
% Dodanie wielu autorów i tłumaczenia na angielski - Kuba Pochrybniak, 29.11.2016

% dodaj opcję [licencjacka] dla pracy licencjackiej
% dodaj opcję [en] dla wersji angielskiej (mogą być obie: [licencjacka,en])
\documentclass[licencjacka,en]{pracamgr}
\usepackage{hyperref}
\usepackage[dvipsnames]{xcolor}
\usepackage{amsfonts}
\usepackage{amsmath}
\usepackage{amsthm}

\definecolor{dark-red}{HTML}{B30000}
\definecolor{dark-green}{HTML}{00B300}
\definecolor{dark-blue}{HTML}{0000B3}

\hypersetup{
    colorlinks=true,
    linkcolor=dark-red,
    urlcolor=dark-blue,
    citecolor=dark-green
}
\urlstyle{same}

% Dane magistranta:
\autor{Marcin Serwin}{1144988}

% Dane magistrantów:
%\autor{Autor Zerowy}{342007}
%\autori{Autor Pierwszy}{342013}
%\autorii{Drugi Autor-Z-Rzędu}{231023}
%\autoriii{Trzeci z Autorów}{777321}
%\autoriv{Autor nr Cztery}{432145}
%\autorv{Autor nr Pięć}{342011}

\title{Overview of cographs related algorithms}
\titlepl{Przegląd algorytmów operujących na kografach}

%\tytulang{An implementation of a difference blabalizer based on the theory of $\sigma$ -- $\rho$ phetors}

%kierunek: 
% - matematyka, informacyka, ...
% - Mathematics, Computer Science, ...
\kierunek{Computer Science}

% informatyka - nie okreslamy zakresu (opcja zakomentowana)
% matematyka - zakres moze pozostac nieokreslony,
% a jesli ma byc okreslony dla pracy mgr,
% to przyjmuje jedna z wartosci:
% {metod matematycznych w finansach}
% {metod matematycznych w ubezpieczeniach}
% {matematyki stosowanej}
% {nauczania matematyki}
% Dla pracy licencjackiej mamy natomiast
% mozliwosc wpisania takiej wartosci zakresu:
% {Jednoczesnych Studiow Ekonomiczno--Matematycznych}

% \zakres{Tu wpisac, jesli trzeba, jedna z opcji podanych wyzej}

% Praca wykonana pod kierunkiem:
% (podać tytuł/stopień imię i nazwisko opiekuna
% Instytut
% ew. Wydział ew. Uczelnia (jeżeli nie MIM UW))
\opiekun{dr inż. Krzysztof Turowski\\
    Jagiellonian University in Cracow\\
}

% miesiąc i~rok:
\date{July 2020}

%Podać dziedzinę wg klasyfikacji Socrates-Erasmus:
\dziedzina{
    %11.0 Matematyka, Informatyka:\\ 
    %11.1 Matematyka\\ 
    % 11.2 Statystyka\\ 
    % 11.3 Informatyka\\ 
    11.3 Informatics, Computer Science \\
    %11.4 Sztuczna inteligencja\\ 
    %11.5 Nauki aktuarialne\\
    %11.9 Inne nauki matematyczne i informatyczne
}

%Klasyfikacja tematyczna wedlug AMS (matematyka) lub ACM (informatyka)
\klasyfikacja{Theory of computation\\
    Design and analysis of algorithms\\
    Graph algorithms analysis\\
    Dynamic graph algorithms}


% TODO: Fix acmart
% \begin{CCSXML}
%   <ccs2012>
%   <concept>
%   <concept_id>10003752.10003809.10003635.10010038</concept_id>
%   <concept_desc>Theory of computation~Dynamic graph algorithms</concept_desc>
%   <concept_significance>300</concept_significance>
%   </concept>
%   </ccs2012>
%   \end{CCSXML}

%   \ccsdesc[300]{Theory of computation~Dynamic graph algorithms}


% Słowa kluczowe:
\keywords{cographs, cograph recognition, maximum clique,
    graph coloring, graph colouring, maximum independent set,
    minimum path cover}

% Tu jest dobre miejsce na Twoje własne makra i~środowiska:
\newtheorem{defi}{Definition}[section]
\newtheorem{observe}{Observation}[section]

\newcommand{\strong}[1]{\textbf{#1}}

\newcommand{\Nats}{\mathbb{B}}

\newcommand{\Cographs}{\mathcal{C}}
\newcommand{\Vertices}{\mathcal{V}}
\newcommand{\singleton}[1]{\text{\textcircled{$v$}}}

\DeclareMathOperator{\V}{V}
\DeclareMathOperator{\E}{E}
\DeclareMathOperator{\C}{C}
\DeclareMathOperator{\R}{R}
\DeclareMathOperator{\Root}{r}
\DeclareMathOperator{\G}{G}
\DeclareMathOperator{\id}{v}
\DeclareMathOperator{\kind}{\mathcal{K}}
\DeclareMathOperator{\LCA}{LCA}

\newcommand{\Gcomp}[1]{\overline{#1}}
\newcommand{\Gunion}[2]{#1 \cup #2}
\newcommand{\BigGunion}[1]{\bigcup #1}
\newcommand{\Gjoin}[2]{#1 \otimes #2}
\newcommand{\BigGjoin}[1]{\bigotimes #1}

% koniec definicji
