\chapter{Implementation and experiments}

All aforementioned algorithms were implemented in Python and are available at \url{https://github.com/marcin-serwin/cographs}. In this chapter we will compare the experimental execution times of cograph recognition algorithms and compare algorithms operating on cotrees to brute force algorithms for general graphs. Times are presented as average over a sample of 100 graphs, and worst time in the same sample. All times are given in milliseconds.

\section{Cograph recognition}
These tests are split into categories by size - 16, 128 and 1024 vertices - and by whether they are cographs or not.

\begin{center}

    \begin{tabular}{ |c|c|c|c|c|}
        \hline
        \multicolumn{5}{|c|}{Cograph recognition - positive answer}                                                 \\
        \hline
                                                                      & number of vertices & 16   & 128   & 1024    \\
        \hline
        \multirow{2}{*}{Brute force algorithm (\ref{20.5-brute})}     & mean               & 0.14 & 12.76 & 191.47  \\
                                                                      & max                & 0.22 & 46.99 & 670.38  \\
        \hline
        \multirow{2}{*}{\cite{habib} algorithm (\ref{21-habib})}      & mean               & 0.43 & 46.91 & 707.28  \\
                                                                      & max                & 0.77 & 96.58 & 1603.73 \\
        \hline
        \multirow{2}{*}{\cite{corneil} algorithm (\ref{22-corneil}) } & mean               & 0.25 & 31.09 & 477.10  \\
                                                                      & max                & 0.63 & 60.93 & 1617.75 \\
        \hline
    \end{tabular}
\end{center}

\begin{center}

    \begin{tabular}{ |c|c|c|c|c|}
        \hline
        \multicolumn{5}{|c|}{Cograph recognition - negative answer}                                                \\
        \hline
                                                                     & number of vertices & 16   & 128   & 1024    \\
        \hline
        \multirow{2}{*}{Brute force algorithm (\ref{20.5-brute})}    & mean               & 0.15 & 15.11 & 283.30  \\
                                                                     & max                & 0.33 & 21.92 & 524.02  \\
        \hline
        \multirow{2}{*}{\cite{habib} algorithm (\ref{21-habib})}     & mean               & 0.41 & 46.25 & 808.24  \\
                                                                     & max                & 0.73 & 80.96 & 1360.41 \\
        \hline
        \multirow{2}{*}{\cite{corneil} algorithm (\ref{22-corneil})} & mean               & 0.10 & 0.33  & 1.03    \\
                                                                     & max                & 0.39 & 1.10  & 1.78    \\

        \hline
    \end{tabular}
\end{center}

As can be easily noticed there isn't much different brute force and efficient algorithms in case of positive answers. In fact the efficient algorithms seem to perform worse. This is caused by the fact that during implementation Python sets were used to simplify the code. They are used much more frequently and extensively in main algorithms when compared to brute force and hence the much worse performance.

Worth noting is also much better performance of \cite{corneil} algorithm in case of negative answer. This is caused by the fact that algorithm is incremental and trying build cotrees one vertex at a time. In case of random graphs it is quite easy to find $P_4$ within the first few chosen vertices. In contrast factorizing permutation algorithms must always process all vertices before checking whether graph is a cograph.

\section{Cotree algorithms}

The tests are split into categories by size - 4, 8, 16, 128 and 1024  vertices. Brute force algorithms are only presented on graphs of size 4, 8 and, when feasible, 16. Since they are exponential it is not feasible to evaluate them for bigger graphs.

Brute force algorithms are naive algorithms that check every potential solution for a problem and finish when they find a correct one. For example in maximal clique problem, brute force algorithm would first check all sets of size $\abs{V}$, then of size $\abs{V}-1$ and so on, until it found set that is a clique.

\begin{center}
    \begin{tabular}{|c|c|c|c|c|c|c|}
        \hline
        \multicolumn{7}{|c|}{Maximal clique}                                                                          \\
        \hline
                                                          & number of vertices & 4    & 8    & 16    & 256   & 1024   \\
        \hline

        \multirow{2}{*}{Brute force}                      & mean               & 0.01 & 0.10 & 40.74 & -     & -      \\
                                                          & max                & 0.02 & 0.15 & 59.51 & -     & -      \\
        \hline
        \multirow{2}{*}{Cotree algorithm (\ref{32-cliq})} & mean               & 0.07 & 0.11 & 0.45  & 52.28 & 546.18 \\
                                                          & max                & 0.22 & 0.12 & 0.93  & 97.77 & 828.48 \\
        \hline
    \end{tabular}
\end{center}


\begin{center}
    \begin{tabular}{|c|c|c|c|c|c|c|}
        \hline
        \multicolumn{7}{|c|}{Maximal independent set}                                                                 \\
        \hline
                                                          & number of vertices & 4    & 8    & 16    & 256   & 1024   \\
        \hline

        \multirow{2}{*}{Brute force}                      & mean               & 0.00 & 0.01 & 48.65 & -     & -      \\
                                                          & max                & 0.10 & 0.15 & 80.67 & -     & -      \\
        \hline
        \multirow{2}{*}{Cotree algorithm (\ref{32-cliq})} & mean               & 0.03 & 0.07 & 0.22  & 29.52 & 532.16 \\
                                                          & max                & 0.05 & 0.10 & 0.28  & 51.92 & 819.60 \\
        \hline
    \end{tabular}
\end{center}


\begin{center}
    \begin{tabular}{|c|c|c|c|c|c|c|}
        \hline
        \multicolumn{7}{|c|}{Coloring}                                                                                   \\
        \hline
                                                              & number of vertices & 4    & 8    & 16   & 256   & 1024   \\
        \hline

        \multirow{2}{*}{Brute force}                          & mean               & 0.02 & 0.80 & -    & -     & -      \\
                                                              & max                & 0.03 & 1.70 & -    & -     & -      \\
        \hline
        \multirow{2}{*}{Cotree algorithm (\ref{32-coloring})} & mean               & 0.04 & 0.08 & 0.22 & 29.47 & 522.12 \\
                                                              & max                & 0.06 & 0.10 & 0.30 & 51.78 & 825.64 \\
        \hline
    \end{tabular}
\end{center}

\begin{center}
    \begin{tabular}{|c|c|c|c|c|c|c|}
        \hline
        \multicolumn{7}{|c|}{Minimal path cover}                                                                       \\
        \hline
                                                           & number of vertices & 4    & 8     & 16   & 256   & 1024   \\
        \hline

        \multirow{2}{*}{Brute force}                       & mean               & 0.03 & 15.41 & -    & -     & -      \\
                                                           & max                & 0.05 & 28.52 & -    & -     & -      \\
        \hline
        \multirow{2}{*}{Cotree algorithm (\ref{33-paths})} & mean               & 0.06 & 0.12  & 0.32 & 30.17 & 535.05 \\
                                                           & max                & 0.09 & 0.15  & 0.48 & 51.92 & 826.64 \\
        \hline
    \end{tabular}
\end{center}
