% !TEX root = ../main.tex

\chapter*{Introduction}
\addcontentsline{toc}{chapter}{Introduction}

Cographs are a special kind of graphs which can be obtained through repeated application of and disjoint union and joint union to a single vertex graph. Several algorithms can be computed efficiently in case of cographs even if no general case algorithm is known.

This paper presents implementation of two algorithms that check whether the input graph is a cograph. The first one is from \cite{habib} and the second one is from \cite{corneil}. The second algorithm generates a cotree of a graph which is then used in order to efficiently implement algorithms for maximum clique, maximum independent set, hamiltonian  path and optimal graph coloring.

Chapter \ref{r:pre} will define necessary concepts used in this paper. Chapter \ref{r:proofs} will prove equivalence of some definitions and correctness of implemented algorithms. Chapter \ref{r:impl} will document usage of implemented algorithms. Implementations of aforementioned algorithms can be found here: \url{https://github.com/marcin-serwin/cographs}.

