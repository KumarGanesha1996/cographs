% !TEX root = ../main.tex

\section{Brute force algorithm}
\label{20.5-brute}

Before we proceed to the main algorithms we will present brute force algorithm described in \cite{habib}. This algorithm uses similar theories and definitions to the main \cite{habib} algorithm so for description of those refer to \ref{21-habib}. Full pseudo code is presented in \ref{brute:main}.

\begin{function}
    \caption{BruteForcePermutation($G$)}
    \label{brute:main}
    \DontPrintSemicolon

    \KwIn{A graph $G = (V,E)$.}
    \KwOut{Factorizing permutation of $G$.}
    \Begin{
        $\Part P = (V)$ \;
        \While{$\exists_{\Part X_i \in \Part P} |\Part X_i| > 1$}{
            $\Part P\colon (\Part X_1, \ldots, \Part X_n)$ \;
            $\Part X\colon \Part X \in \Part P \land \abs{\Part X} > 1$\;
            $x \colon x \in \Part X$\;
            $L = \Part X \cap \Nbar(x)$\;
            $R = \Part X \cap \N(x)$\;

            \ForEach{$y \in \Part X \cap \Nbar(x), \Part X' \in R$} {
                $R$.replace($\Part X'$).with($\Part X' \cap \Nbar(y), \Part X' \cap \N(y)$) \;
            }
            \ForEach{$y \in \Part X \cap \N(x), \Part X' \in L$} {
                $L$.replace($\Part X'$).with($\Part X' \cap \Nbar(y), \Part X' \cap \N(y)$) \;
            }

            $\Part P = (\Part X_1, \ldots L, \{x\}, R, \ldots, \Part X_n)$
        }

        \Return{$\Part P$}
    }
\end{function}

Each time the inside of the while loop is executed we create at least one singleton part is added to permutation. Since there may be at most $\abs{V}$ singleton it is clear that the while loop will execute at most $O(\abs{V})$ times. Each loop execution requires looking through potentially $O(\abs{V})$ vertices in order to apply the second refinement rule. Therefore the whole algorithm takes up $O(\abs{V}^2)$.
