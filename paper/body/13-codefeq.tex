% !TEX root = ../main.tex

\section{Cograph definitions equivalence}\label{r:codefeq}

In this section we will prove that all five aforementioned definitions of graphs are equivalent. Next sections will provide proofs for, in order, \ref{codef1} $\implies$ \ref{codef2}, \ref{codef2} $\implies$ \ref{codef3}, \ref{codef3} $\implies$ \ref{codef4}, \ref{codef4} $\implies$ \ref{codef5} and \ref{codef5} $\implies$ \ref{codef1}. These five implications together prove the equivalence.

\subsection{Definition \ref{codef1} $\implies$ definition \ref{codef2}}

Let us notice that in order to prove the implication all we need to show is $\forall_G G \in \Cographs \implies \Gcomp{G} \in \Cographs$ where $\Cographs$ is defined as in \ref{codef2}.

Let's assume the opposite and let $G$ be the smallest graph such that $G \in \Cographs \land \Gcomp{G} \not\in \Cographs$. Since $G \in \Cographs$ one of the following must be true:
\begin{enumerate}
    \item $\exists_v G = \singleton{v}$. But $\Gcomp{\mbox{\singleton{v}}\raisebox{3mm}{}} = \singleton{v}$, so $\Gcomp{G} \in \Cographs$, therefore it cannot be so.
    \item $\exists_{G_1, G_2 \in \Cographs} G = \Gunion{G_1}{G_2}$. But since $G$ is the smallest graph that violates the thesis and $G_1$ and $G_2$ are both smaller, it follows that $\Gcomp{G_1}, \Gcomp{G_2} \in \Cographs$. On the other hand it is easy to see that $\Gcomp{\Gunion{G_1}{G_2}} = \Gjoin{\Gcomp{G_1}}{\Gcomp{G_2}}$, so $\Gcomp{G} \in \Cographs$, therefore it cannot be so.
    \item $\exists_{G_1, G_2 \in \Cographs} G = \Gjoin{G_1}{G_2}$. But since $G$ is the smallest graph that violates the thesis and $G_1$ and $G_2$ are both smaller, it follows that $\Gcomp{G_1}, \Gcomp{G_2} \in \Cographs$. On the other hand it is easy to see that $\Gcomp{\Gjoin{G_1}{G_2}} = \Gunion{\Gcomp{G_1}}{\Gcomp{G_2}}$, so $\Gcomp{G} \in \Cographs$, therefore it cannot be so.
\end{enumerate}
Since one of these must be true and none of them is, it follows that such $G$ does not exist and $\forall_G G \in \Cographs \implies \Gcomp{G} \in \Cographs$.

\subsection{Definition \ref{codef2} $\implies$ definition \ref{codef3}}

In order to prove the implication it is enough to show that for every graph in $\Cographs$, as defined in \ref{codef2}, there exists a cotree that defines it. Let us show that it is indeed the case

\begin{enumerate}
    \item $v \in \Vertices \implies \singleton{v} \in \Cographs$

          It is indeed true that that $\singleton{v}$ has a cotree. Let $0_v, 1_v \in \Vertices \setminus \{v\}$. Now
          \[
              T = ((\{0_v,1_v,v\}, \{(1_v,\bot), (0_v,1_v),(v,0_v)\}), \{(0_v,0), (1_v,1), (v,s)\})
          \]
          is such that $\G(T) = \singleton{v}$.

    \item $G_1,G_2 \in \Cographs \implies \Gunion{G_1}{G_2} \in \Cographs$

          We are inductively assuming that there exist cotrees for $G_1$ and $G_2$, respectively $T_1 = ((V_1, p_1), \kind_1)$ and $T_2 = ((V_2, p_2), \kind_2)$. Moreover let $r_1$ and $r_2$ be respectively $\Root(T_1)$ and $\Root(T_2)$.

          Let $0_v, 1_v \in \Vertices \setminus (V_1 \cup V_2)$. Now let us construct a cotree $T = ((V, p), \kind)$ such that
          \[
              \begin{split}
                  V = & V_1\cup V_2 \cup \{0_v,1_v\}, \\
                  p = & p_1 \cup p_2 \setminus \{(r_1,\bot), (r_2, \bot)\} \cup \{(1_v, \bot), (0_v, 1_v), (r_1, 0_v), (r_2, 0_v)\}
                  ), \\
                  \kind = & \kind_1 \cup \kind_2 \cup \{(0_v,0), (1_v,1)\}
              \end{split}
          \]

          Notice that the set of vertices such that $\kind(v) = s$ is the sum of of sets such that $\kind_1(v) = s$ and $\kind_2(v) = s$.

          Let us consider a pair of vertices $v_1, v_2 \in V_i$ the $\kind(\LCA_T(v_1, v_2)) = \kind_{i}(\LCA_{T_{i}}(v_1, v_2))$ and so edges between them will be preserved in graph defined by $T$ for $i \in \{1,2\}$.

          Now let us consider a pair of vertices $v_1 \in V_i, v_2 \in V_j$ where $i,j \in \{1,2\} \land i \neq j$. The $\LCA_T(v_1, v_2) = 0_v$ and so $\kind(\LCA_T(v_1, v_2)) = 0$. This means that there are no edges between vertices from different graphs.

          These three facts together show that $\G(T)$ is such that $\V(\G(T)) = V_1 \cup V_2, \G(T)[V_1] = G_1, \G(T)[V_2] = G_2$ and $\E(\G(T)) \cap \{\{v_1, v_2\} : (v_1, v_2) \in V_1 \times V_2\} = \emptyset$ which means that $\G(T) = \Gunion{G_1}{G_2}$.

    \item $G_1,G_2 \in \Cographs \implies \Gjoin{G_1}{G_2} \in \Cographs$

          We are inductively assuming that there exist cotrees for $G_1$ and $G_2$, respectively $T_1 = ((V_1, p_1), \kind_1)$ and $T_2 = ((V_2, p_2), \kind_2)$. Moreover let $r_1$ and $r_2$ be respectively $\Root(T_1)$ and $\Root(T_2)$.

          Let $0_v, 0_v', 1_v \in \Vertices \setminus (V_1 \cup V_2)$. Now let us construct a cotree $T = ((V, p), \kind)$ such that
          \[
              \begin{split}
                  V = & V_1\cup V_2 \cup \{0_v,0_v',1_v\}, \\
                  p = & p_1 \cup p_2 \setminus \{(r_1,\bot), (r_2, \bot)\} \cup \{(1_v, \bot), (0_v, 1_v), (0_v', 1_v), (r_1, 0_v), (r_2, 0_v')\}
                  ), \\
                  \kind = & \kind_1 \cup \kind_2 \cup \{(0_v,0), (0_v',0), (1_v,1)\} \\
              \end{split}
          \]

          It is easy to verify, using process analogous to the previous case, that $\G(T) = \Gjoin{G_1}{G_2}$.

          %   Notice that the set of vertices such that $\kind(v) = s$ is $V_1 \cup V_2$.

          %   Let us consider a pair of vertices $v_1, v_2 \in V_i$ the $\kind(\LCA_T(v_1, v_2)) = \kind_{i}(\LCA_{T_{i}}(v_1, v_2))$ and so edges between them will be preserved in graph defined by $T$ for $i \in \{1,2\}$.

          %   Now let us consider a pair of vertices $v_1 \in V_i, v_2 \in V_j$ where $i,j \in \{1,2\} \land i \neq j$. The $\LCA_T(v_1, v_2) = 1_v$ and so $\kind(\LCA_T(v_1, v_2)) = 1$. This means that there are always edges between vertices from different graphs.

          %   These three facts together show that $\G(T)$ is such that $\V(\G(T)) = V_1 \cup V_2, \G(T)[V_1] = G_1, \G(T)[V_2] = G_2$ and $\E(\G(T)) \cap \{\{v_1, v_2\} | (v_1, v_2) \in V_1 \times V_2\} = \{\{v_1, v_2\} | (v_1, v_2) \in V_1 \times V_2\}$ which means that $\G(T) = \Gjoin{G_1}{G_2}$.

\end{enumerate}

\subsection{Definition \ref{codef3} $\implies$ definition \ref{codef4}}

Let's assume the opposite and let $T$ be a cotree such that $G = \G(T)[X]$ is isomorphic to $P_4$ for some $X \subseteq \V(\G(T))$. Now let's notice that if $v = \LCA_T^s(X)$ then it must be so that $\kind(v) = 1$, otherwise $G$ wouldn't be isomorphic to $P_4$ as it isn't connected.

Let $X = \{x_1, x_2, x_3, x_4\}$ and let $v_1, v_2, v_3, v_4$ be (not necessarily different) vertices of $T$ such that $\forall_{i \in \{1,2,3,4\}} p_T(v_i) = v \land x_i \in \V(T[v_i])$. In other words, $v_i$ is a child of $v$ that contains $x_i$ in their subtree. They are not necessarily different since some $x_j$ may be in the same child's subtree as $x_i$. Now one of 4 cases must be true
\begin{enumerate}
    \item $|\{v_1, v_2, v_3, v_4\}| = 4$, or all are different.

          In this case for all $i,j \in \{1,2,3,4\}$ it's true that $\LCA_T(x_i, x_j) = v$. Therefore \linebreak $\forall_{i,j \in \{1,2,3,4\}}\{x_i,x_j\}\in \E(G)$ and $G = (X, \binom{X}{2})$. There is no isomorphism between $G$ and $P_4$ because $|\E(G)| > |\E(P_4)|$.

    \item $|\{v_1, v_2, v_3, v_4\}| = 3$, without loss of generality we assume that $v_1 = v_2$ and all other are different.

          In this case we know that $\{\{x_3, x_4\}, \{x_1, x_3\}, \{x_1, x_4\}, \{x_2, x_3\}\} \subseteq \E(G)$, which means that $|\E(G)| \geq 4$ which means that $|\E(G)| > |\E(P_4)|$.


    \item $|\{v_1, v_2, v_3, v_4\}| = 2$ and the split is $3$ to $1$. We assume that $v_1 = v_2 = v_3 \neq v_4$.

          In this case we know that $\{\{x_1, x_4\}, \{x_2, x_4\}, \{x_3, x_4\}\} \subseteq \E(G)$. Let us assume that there exists isomorphism $f$ from $G$ to $P_4$. It follows that there exists vertex $f(x_4) \in \V(P_4)$ that has edges to all 3 vertices $f(x_1), f(x_2), f(x_3)$. But such vertex does not exist and so $G$ and $P_4$ are not isomorphic.

    \item $|\{v_1, v_2, v_3, v_4\}| = 2$ and the split is $2$ to $2$. We assume that $v_1 = v_2 \neq v_3 = v_4$.

          In this case we know that $\{\{x_1, x_3\}, \{x_1, x_4\}, \{x_2, x_3\}, \{x_2, x_4\}\} \subseteq \E(G)$, which means that $|\E(G)| \geq 4$ which means that $|\E(G)| > |\E(P_4)|$.
\end{enumerate}

In all four cases $G$ and $P_4$ are not isomorphic implies that in no case can $G$ be isomorphic to $P_4$ and so if graph is defined by a cotree then it has no subgraph isomorphic to $P_4$.

\subsection{Definition \ref{codef4} $\implies$ definition \ref{codef5}}

Let's assume the contrary and that there are such graphs that contradicts this implication. Let $G$ be the smallest of such graphs. Now let $v \in \V(G)$ and $G' = G[\V(G) \setminus \{v\}]$. Since $G'$ is smaller than $G$, it is true that either $G'$ or $\Gcomp{G'}$ is disconnected. Without loss of generality let's assume that $G'$ is disconnected (the other case can be resolved by starting with $\Gcomp{G}$ as the smallest counterexample).

Let $X_1, \ldots, X_n$ be such that $\forall_{i \in \{1, \ldots, n\}} G[X_i]$ is connected and $\forall_{x \in \V(G) \setminus X_i} G[X_i \cup \{x\}]$ is disconnected. Now it must be true that $v$ has edge to at least one vertex in each $X_i$, otherwise $G$ would be disconnected. Let's assume that $\forall_{i \in \{1,\ldots,n\}} \forall_{x \in X_i} \{v,x\} \in \E(G)$. This would mean that $\Gcomp{G'}$ is disconnected since $v$ is an isolated vertex. This means that there exists $i$ such that $\exists_{u, u' \in X_i} \{v, u\} \in \E(G) \land \{v, u'\} \not\in \E(G)$. Without loss of generality let's assume that $i = 1$.

Let us notice that there must be two vertices $x, y \in X_1$ such that $\{v, x\}, \{x, y\} \in \E(G)$ and $\{v,y\} \not\in \E(G)$. If there was no such vertices then the vertex that has an edge to $v$ would not be connected to vertex that has no edge, hence $G[X_1]$ would be disconnected. Now let $z \in X_2$ be such that $\{v, z\} \in \E(G)$. We claim that $G[\{x,y,z,v\}]$ is isomorphic to $P_4$.

Indeed $G[\{x,y,z,v\}] = (\{x,y,z,v\}, \{\{v,x\}, \{x, y\}, \{v,z\}\})$ and after applying function $f = \{(y, 0), (x, 1), (v, 2), (z, 3)\}$ we get $(\{1,0,3,2\}, \{\{2,1\}, \{1, 0\}, \{2,3\}\})$ which is indeed $P_4$.

\subsection{Definition \ref{codef5} $\implies$ definition \ref{codef1}}

Let $G$ be the smallest graph that contradicts the implication.

Let $G$ be disconnected. Since $G$ is the smallest counterexample then all maximal connected subgraphs $G[X_1], \ldots, G[X_n]$ must be in $\Cographs$. But then $\Gunion{G[X_1]}{\ldots} \Gunion{}{G[X_n]} = G \in \Cographs$. This means that $G$ cannot be connected.

Let $\Gcomp{G}$ be disconnected. Since $G$ is the smallest counterexample then all maximal connected subgraphs $\Gcomp{G[X_1]}, \ldots, \Gcomp{G[X_n]}$ must be in $\Cographs$. But then $\Gcomp{\Gunion{\Gcomp{G[X_1]}}{\ldots} \Gunion{}{\Gcomp{G[X_n]}}} = G \in \Cographs$. This means that $G$ cannot be disconnected.

Since either $G$ or $\Gcomp{G}$ must be disconnected but none is, it follows that there is no graph that contradicts the implication.
