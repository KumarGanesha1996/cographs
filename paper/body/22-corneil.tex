\section{Generating cotree algorithm}
\label{22-corneil}


The algorithm, described fully in \cite{corneil}, is incremental in the sense that vertices are processed one at a time. Formally this means that given a cotree for a graph $G' = (V \setminus \{v\}, E \setminus \{\{u,v\} : u \in V\})$, the algorithm modifies cotree so that it is a valid cotree for a graph $G = (V,E)$ (if it is possible).

\subsection{"Marking" vertices}

Before proceeding to the main loop of the algorithm let us present a \ref{corneil:mark} function which will be used therein. For any node $v$ of cotree $T$, $v$.md denotes the number of $v$ which were marked or unmarked. Initially this value is $0$ and it is set to $0$ when $v$ is unmarked.


\begin{function}
    \caption{Mark($G, T, v$)}
    \label{corneil:mark}
    \DontPrintSemicolon

    \KwIn{Graph $G=(V,E)$, cotree $T$ which defines graph $G[X]$ for some $X$ and vertex $v \not \in X$.}
    \KwOut{Set of marked vertices}
    \Begin{
        $M = \N(v) \cap X$
        $Q = M$

        \While{$\abs{Q} > 0$}{
            $v = Q$.pop() \;
            $M$.remove($v$) \;
            $v$.md $= 0$ \;
            $p = p_T(v)$ \;
            $M$.insert($p$) \;
            $p$.md $+= 1$ \;
            \If{$p$.md = $\abs{\C_T(p)}$}{
                $Q$.append($p$) \;
            }
        }
        \If{$\abs{M} > 0 \land \abs{\C(\Root(T))} = 1$}{
            $M$.insert($\Root(T)$) \;
        }
        \Return{$M$}
    }
\end{function}

Let $M$ be the result of the function \ref{corneil:mark}. This result alone is enough to determine whether the graph $G[X \cup \{v\}]$ is a cograph. To do so we must introduce some definitions.

\begin{defi}
    If $v \in M$ and $\kind_T(v) = 1$, we say that $v$ is \emph{properly marked} if $v$.md = $\abs{\C(v)} -1$.
\end{defi}

\begin{defi}
    If $T$ is a marked cotree, then a path $(v_1, \ldots, v_n)$ is a \emph{legitimate alternating path} if $v_1, v_n$ are properly marked nodes and $v_2, v_4, \ldots, v_{n-1}$ are unmarked nodes such that $\kind(v_{2k}) = 0$.
\end{defi}

With these definitions we can present the main lemma used to identify cographs.

\begin{lemma}
    \label{corneil:mainlemma}
    Let $\alpha$ be a node in $M$ in the lowest level of cotree and let $\beta$ be a node in $M \setminus \{\alpha\}$ in the lowest level of cotree. If $G[X]$ is a cograph then $G[X \cup \{v\}]$ is a cograph if and only if either $M = \emptyset$ or
    \begin{enumerate}
        \item $M \setminus \{\alpha\}$ consists only of properly marked nodes of a (possibly empty) legitimate alternating path which ends at $\Root(T)$ and
        \item $\alpha$ is such that $\kind_T(\alpha) = 0$ and $p_T(\alpha) = \beta$ or $\alpha$ is such that $\kind_T(\alpha) = 1$ and $p_T(p_T(\alpha)) = \beta$.
    \end{enumerate}
\end{lemma}

Proof of \ref{corneil:mainlemma} can be found in \cite{corneil}.

The \ref{corneil:mark} function will only examine $O(\abs{\N(v)})$ vertices. It is easy to see since initially at most $\abs{\N(v)}$ are marked and since all internal nodes (except root) have at least two children, pushing upward will be decreasing the number of marked vertices.

\subsection{Cograph recognition}

Using the \ref{corneil:mark} we can implement the main cotree constructing algorithm. As was mentioned this algorithm is incremental we start with a cotree with two vertices and then add next vertices until cotree for a full graph is constructed. Algorithm is stopped if adding another vertex the graph stops being a cograph. Algorithm is fully described below in \ref{corneil:main}.

\begin{function}
    \caption{ConstructCotree($G$)}
    \label{corneil:main}
    \DontPrintSemicolon
    \SetKwFunction{FMark}{Mark}
    \SetKwFunction{FLowest}{FindLowest}

    \KwIn{Graph $G=(V,E)$.}
    \KwOut{Cotree $T$ such that $\G(T) = G$, if $G$ is a cograph, $\bot$ otherwise.}
    \Begin{
        $v_1, v_2 \colon v_1, v_2 \in V$ \;
        $T =$ cotree with a single (1) vertex as a root \;
        \lIf{$\{v_1, v_2\} \in E$}{
            $\Root(T)$.add\_children($v_1, v_2$)
        }\lElse{
            $\Root(T)$.add\_children(new (0) node with $v_1, v_2$ as children
        }

        \ForEach{$v \in V\setminus\{v_1, v_2\}$}{
            $M =$ \FMark($G,T,v$) \;
            \uIf{all nodes of $T$ were marked and unmarked}{
                $\Root(T)$.add\_children($v$)\;
            }
            \uElseIf{no nodes were marked}{
                \eIf{$\abs{\C(\Root(T))} = 1$}{
                    $c\colon c \in \C(\Root(T))$\;
                    $c$.add\_children($v$)\;
                }{
                    $c =$ new (0) node with $\Root(T)$ and $v$ as a children \;
                    $T =$ cotree with (new (1) node with $c$ as a child) as a root\;
                }
            }
            \Else{
                $u =$ \FLowest($T, M$)\;
                \lIf{$u = \bot$}{\Return{$\bot$}}
                $C = $ \leIf{$\kind(u) = 1$}{$\{v:v\in \C(u) \land v$ were marked$\}$}{$\{v:v\in \C(u) \land v$ were not marked$\}$}

                \eIf{$\abs{C}= 1$}{
                    \eIf{$w \in C \land \kind(w) = s$}
                    {
                        $c =$ new ($1-\kind(u)$) node with $w, v$ as children \;
                        replace $w$ with $c$ as a new $u$ child \;
                    }
                    {$w$.add\_children($v$)}
                }{
                    $y =$ new ($\kind(u)$) node with $\{v:v\in \C(u) \land v$ were marked$\}$ as children\;
                    $\C(u) = \C(u) \setminus \{v:v\in \C(u) \land v$ were marked$\}$ \;
                    \eIf{$\kind(u) = 0$}{
                        $u$.add\_children(new (1) node with $v,y$ as children)
                    }{
                        replace $u$ with $y$ as a new $p_T(u)$ child \;
                        $y$.add\_children(new (0) node with $v,u$ as children)
                    }
                }
            }
        }
        \Return{$T$}
    }
\end{function}

Within \ref{corneil:main} function a \ref{corneil:findlowest} is used. This function checks whether the graph after adding a new vertex is still a cograph and if it is it returns the lowest marked vertex. The function goes through marked vertices and checks whether they are on a legitimate alternating path. It unmarks the vertices along this paths so that we always choose a lower one in new iteration. If a necessary condition fails it stops and returns $\bot$, otherwise the main loop is repeated until no marked vertices remain.

\begin{function}
    \caption{FindLowest($T, M$)}
    \label{corneil:findlowest}
    \DontPrintSemicolon

    \KwIn{Cotree $T$ and marked vertices $M$.}
    \KwOut{Lowest marked vertex of cotree $T$ if $G[X \cup v]$ is a cograph, $\bot$ otherwise.}
    \Begin{
        $y = \top$ \;
        \lIf{$\Root(T) \not\in M$}{\Return{$\bot$}}
        \If{$\abs{\C(\Root(T))} - 1 \neq \Root(T)$.md}{
            $y = \bot$ \;
            $M = M \setminus \Root(T)$ \;
            $\Root(T)$.md = 0 \;
            $u = w = \Root(T)$ \;
        }

        \While{$M \neq \emptyset$}{
            $u = M$.pop() \;
            \lIf{$y \neq \top$}{\Return{$\bot$}}
            \eIf{$\kind(u) = 1$}{
                \lIf{$u$.md $\neq \abs{\C(u)} - 1$}{$y = \bot$}
                \lIf{$p_T(u) \in M$}{\Return{$\bot$}}
                $t = p_T(p_T(u))$ \;
            }{
                $y = \bot$\;
                $t = p_T(u)$
            }
            $M = M \setminus \{u\}$ \;
            $u$.md = 0

            \While{$t \neq w$}{
                \lIf{$t = \Root(T)$}{\Return{$\bot$}}
                \lIf{$t \not\in M$}{\Return{$\bot$}}
                \lIf{$t$.md $\neq \abs{\C(t)}$}{\Return{$\bot$}}
                \lIf{$p_T(t) \in M$}{\Return{$\bot$}}
                $M = M \setminus \{t\}$ \;
                $t$.md = 0 \;
                $t = p_T(p_T(t))$ \;
            }
            $w = u$
        }
        \Return{$u$}
    }
\end{function}

Almost all tree alterations can be done in $O(1)$ time. The only exception is when the lowest marked node has two or more children which have been marked and unmarked, or not marked (depending of the kind of node). Since the set of marked and unmarked is bounded by $O(\abs{\N(v)})$ (as it was produced by \ref{corneil:mark} function), this operation can be done in $O(\abs{\N(v)})$.
Combining this fact with $O(\abs{\N(v)})$ timing of \ref{corneil:mark} function it is easy to conclude that the whole algorithm runs in $O(\sum_{v \in V} \abs{\N(v)}) = O(\abs{V} + \abs{E})$ time.