% !TEX root = ../main.tex

\chapter{Algorithms implemented}\label{r:codefeq}

In this chapter we will describe the algorithms implemented in order to recognize the cographs and solve some problems on them.

First cograph recognition algorithm is implemented from \cite{habib}. It produces factorizing permutation if input graph is a cograph and some incorrect permutation otherwise. Algorithm checking whether the permutation is factorizing is then run in order to differentiate between cographs and non-cographs.

Second cograph recognition is implemented from \cite{corneil}. It produces cotree and halts if it is not possible. Cotree produced by this algorithm is then used as an input into the other algorithms solving problems for cotrees.

Implemented are also four algorithms which are NP-hard in general case, but can be solved linearly in case of cographs. These algorithms are: minimal path cover, optimal coloring, finding maximal clique, finding minimal independent set. Last three algorithms follow a very similar pattern of combining recursively results from subcotrees. In case of minimal path cover, implementation described in \cite{olariu}. It is worth mentioning that deciding whether cograph is Hamiltonian (which may be seen as a decision version of minimal path cover problem) can also be done in vein similar to the other algorithms as described in \cite{corneil2}.
