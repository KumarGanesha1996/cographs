% !TEX root = ../main.tex

\section{Terminology and notation}

Fundamental notion in this paper is a graph. All graphs used in this paper are assumed to be \emph{finite undirected graphs} unless specified otherwise. $\Vertices$ is assumed to be vertex universe. Throughout this paper we assume that $\Nats = \Vertices$. Formal definitions follow below.

\begin{defi}
    \emph{Graph} $G$ is a tuple $(V,E)$ such that $V \subseteq \Vertices \land |V| \in \Nats$ and $E \subseteq \binom{V}{2}$. Convenience functions $\V, \E$ are defined such that $\V(G) = V$ and $\E(G) = E$. Set $V$ is called \emph{vertices} and set $E$ is called \emph{edges}.
\end{defi}

\begin{defi}
    \emph{Size of graph $G$} is $|\V(G)|$. Whenever words referring to size are used in context of graphs (e.g. "bigger", "smallest") it refers to the size of graph.
\end{defi}

\begin{defi}
    Let $G = (V,E)$ be a graph and $u, v \in V$. Vertices $u$ and $v$ are \emph{connected} if there exists a sequence of edges $(u_i,v_i)$ for $i \in \{1, \ldots, n\}$ such that:
    \begin{enumerate}
        \item $u_1 = u$
        \item $v_n = v$
        \item $\forall_{i \in \{1, \ldots, n - 1\}} v_i = u_{i+1}$
    \end{enumerate}

    Vertices are \emph{disconnected} if and only if they are not connected.
\end{defi}

\begin{defi}
    Graph $G = (V,E)$ is said to be \emph{connected} if for each pair of vertices $u,v \in V$ $u$ and $v$ are connected.

    Graph is \emph{disconnected} if and only if it is not connected.
\end{defi}

\begin{defi}
    Let $G = (V,E)$ be a graph and $v \in V$. Vertex $v$ is \emph{isolated} if and only if $\forall_{v' \in V} \{v,v'\} \not\in E$.
\end{defi}

\begin{defi}
    If $G = (V, E)$ is a graph then $\Gcomp{G}$ is a \emph{complementary graph}. Formally $\Gcomp{G} = (V, \binom{V}{2} \setminus E)$.
\end{defi}
\begin{defi}
    If $G_1 = (V_1, E_1), G_2 = (V_2, E_2)$ are graphs then $\Gunion{G_1}{G_2}$ is a \emph{union of graphs}. Formally $\Gunion{G_1}{G_2} = (V_1 \cup V_2, E_1 \cup E_2)$.
\end{defi}
\begin{defi}
    If $G_1 = (V_1, E_1), G_2 = (V_2, E_2)$ are graphs then $\Gjoin{G_1}{G_2}$ is a \emph{join of graphs}. Formally $\Gjoin{G_1}{G_2} = (V_1 \cup V_2, E_1 \cup E_2 \cup \{\{v_1,v_2\}|v_1 \in V_1, v_2 \in V_2\})$.
\end{defi}

\begin{defi}
    Graph $\singleton{v}$ is a single vertex graph with vertex $v$. Formally $\singleton{v} = (\{v\}, \emptyset)$.
\end{defi}

\begin{defi}
    If $G = (V,E)$ is a graph and $X \subseteq V$ then $G[X]$ is a \emph{subgraph induced by $X$}. Formally $G[X] = (X, E \cap \binom{X}{2})$.
\end{defi}

\begin{defi}
    Graphs $G_1 = (V_1, E_1), G_2 = (V_2, E_2)$ are said to be isomorphic if there exists bijection $f: V_1 \mapsto V_2$ such that $\forall_{v, v' \in V_1} \{v,v'\} \in V_1 \iff \{f(v), f(v')\} \in E_2$.
\end{defi}

\begin{defi}
    \emph{Tree} $T$ is a tuple $(V, p)$ such that $V \subseteq \Vertices \land |V| \in \Nats$ and $p$ is a function $V \to V \cup \{\bot\}$, where $\bot$ is such that $\bot \not\in V$. Moreover $p$ satisfies
    \begin{enumerate}
        \item $\exists_{r \in V} p(r) = \bot \land \forall_{v \in V\setminus\{r\}} p(v) \neq \bot$. $r$ is called \emph{root} and is denotes as $\Root(T)$.
        \item If $X$ is the smallest set such that $r \in X \land \forall_{v \in V} p(v) \in X \implies v \in X$ then $X = V$.
    \end{enumerate}
    Convenience functions $\V, \R, p_T, \C$ are defined so that $\V(T) = V, \R(T) = r, p_T(v) = p(v)$ and $\C(v) = \{v' | v' \in V \land p(v') = v\}$.
\end{defi}

\begin{defi}
    If $(V,p)$ is a tree and $v \in V$ then \emph{subtree rooted at v} denoted as $T[v]$ is a tree $(V', p')$ such that
    \[
        p'(x) = \begin{cases}
            \bot & \text{if } x = v \\
            p(x) & \text{else}
        \end{cases}
    \]
    and $V'$ is the smallest set such that $v \in V' \land \forall_{v' \in V} p'(v') \in V' \implies v' \in V'$.
\end{defi}

\begin{defi}
    If $T$ is a tree and $u,v \in \V(T)$ then \emph{lowest common ancestor} of $u, v$ or $\LCA_T(u,v)$ is a vertex $l \in \V(T)$ such that $u,v \in \V(T[l]) \land \forall_{l' \in \C(l)} u,v \not\in \V(T[l'])$.
\end{defi}

\begin{defi}
    The function $\LCA_T^s : 2^{\V(T)} \setminus \{ \emptyset \} \mapsto \V(T)$ for $X \subseteq \V(T)$ is defined such that:
    \[
        \LCA_T^s(X) = \begin{cases}
            v                                      & \text{if } X = \{v\}                \\
            \LCA_T(v, \LCA_T^s(X \setminus \{v\})) & \text{if } |X| \geq 2 \land v \in X \\
        \end{cases}
    \]
\end{defi}
