\section{Minimal path cover}

The last algorithm implemented solves the minimal path cover algorithm. The problem asks us to find smallest set of paths $S = \{(v_1^1, \ldots, v_{n_1}^1), \ldots, (v_1^k, \ldots, v_{n_k}^k) \}$ such that
\[
    \begin{split}
        & \{v_1^1, \ldots, v_{n_1}^1, \ldots, v_1^k, \ldots, v_{n_k}^k\} = V \land \\
        & \forall_{i_1,i_2 \in \{1,\ldots,k\},j_1 \in \in \{1, \ldots, n_{i_1}\},j_2 \in \in \{1, \ldots, n_{i_2}\}}  \land \\
        & \forall_{i \in \{1, \ldots, k\}, j \in \{1, \ldots, n_i - 1\}} \{v_j^k\}.
    \end{split}
\]
In other words the objective is to find smallest set of disjoint paths covering the whole graph. The algorithm implemented is fully described in \cite{olariu}.

\subsection{Mend and merge}

The main section of the algorithm consists of \ref{pathcover:mendmerge} function. This function handles (1) nodes of cotree by joining minimum path cover of one subgraph with vertices of the second subgraph. The details are described below.


\begin{function}
    \caption{MendAndMerge($V, S$)}
    \label{pathcover:mendmerge}
    \DontPrintSemicolon
    \SetKwFunction{FSum}{sum}

    \KwIn{Set of vertices $V$ of graph $G_1$ and minimal path cover $S$ of graph $G_2$.}
    \KwOut{Minmial path cover of graph $\Gjoin{G_1}{G_2}$.}
    \Begin{
        $P = S$.pop() \;
        $s =$ \FSum($\{\abs{p} : p \in S\}$) \;
        \While{$V \neq \emptyset \land s \geq \abs{V}$}{
            $P' = S$.pop() \;
            $v = V$.pop() \;
            $P = (P, v, P')$ \;
            $s = s - \abs{P'}$ \;
        }
        \If{$V = \emptyset$}{
            \Return{$S \cup \{P\}$}
        }
        $S' = \{u_1, \ldots, u_k : \forall_{i \in \{1, \ldots, k\}}\exists_{p \in S} u_i \in p\}$ \;
        $P = (v_1, \ldots, v_n)$\;
        $V = \{w_1, \ldots, w_m\}$\;
        \Return{$\{(
                v_1, \ldots, v_{n - m + k},
                w_1, v_{n - m + k + 1},\ldots, v_n,
                w_{m-k}, u_1, w_{m-k + 1} \ldots, u_k, w_m)\}$} \;

    }
\end{function}

The basic idea of the function is to take all vertices of graph $G_1$ and use them to connect the paths from minimal path cover of $G_2$. If the vertices of $G_1$ run out while connecting paths we return the merged paths and the remaining ones as the minimal path cover. Otherwise if the number of unused vertices in $G_1$ drops below the number of vertices in the remaining paths, we construct Hamiltonian path and return it as a minimal path cover. The fact that this Hamiltonian path is a proper path and always exists is proved in Lemma 2 of \cite{olariu}.

If the sets in this algorithm are maintained using linked lists it is easy to see that the first while loop in the algorithm takes at most $O(\abs{V})$ steps. Similarly the second stage comprises of first removing at most $O(\abs{V})$ elements from path $P$ and then extending that path with at most $2\abs{A} - 1 \in O(\abs{V})$ elements. This means that the function \ref{pathcover:mendmerge} is bounded by $O(\abs{V})$.

\subsection{Main algorithm}

In order for this algorithm to work we must ensure that there are always fewer vertices in $V$ than in $S$ when passing it to \ref{pathcover:mendmerge} function. This forces us to process the nodes of the cotree in a certain order. Details can be seen in \ref{pathcover:minpathcover} function.


\begin{function}
    \caption{FindMinPathCover($T$)}
    \label{pathcover:minpathcover}
    \DontPrintSemicolon
    \SetKwFunction{FMain}{FindMinPathCover}
    \SetKwFunction{FMandM}{MendAndMerge}

    \KwIn{Cotree $T = ((V, p), \kind)$.}
    \KwOut{Minmial path cover of graph $\G(T)$.}
    \Begin{
        \lIf{$\kind(\Root(T)) = s$}{\Return{$\{\Root(T)\}$}}

        \If{$\kind(\Root(T)) = 0$}{
            $r = \emptyset$
            \ForEach{$c \in \C(\Root(T))$}{
                $r = r \cup$ \FMain($T[c]$)
            }
            \Return{$r$}
        }

        \lIf{$\C(\Root(T)) = \{v\}$} {\Return{\FMain($T[v]$)}}

        \ForEach{$c \in \C(\Root(T))$}{
            \eIf{$\abs{\V(T[c])} > \abs{\V(T)} - \abs{\V(T[c])}$}{
                \Return{\FMandM($\V(T) \setminus \V(T[c])$, \FMain($T[c]$))}
            }{
                \Return{\FMandM($\V(T[c])$, \FMain($(V \setminus \V(T[c]), p \setminus (p_{T[c]}\cup \{(c, \Root(T))\}), \kind \setminus \kind_{T[c]}$))}
            }
        }
    }
\end{function}

It is easy to see that if sets are maintained as linked lists the (0) nodes are take $O(1)$ time to process. In case of (1) nodes we use \ref{pathcover:mendmerge} function which is bounded by $O(\abs{V})$. But since we make sure that $\abs{V} < \abs{v : \exists{p \in S} v\in p\}}$ at each step it follows that the whole algorithm runs in $O(\abs{\V(T)})$ time.
