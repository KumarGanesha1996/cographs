% !TEX root = ../main.tex

\chapter{Preliminaries}\label{r:pre}

Pojęciem pierwotnym blabalii fetorycznej jest \emph{blaba}.
Blabaliści nie podają jego definicji, mówiąc za Ciach-Pfe t-\=am
K\^un (fooistyczny mędrzec, XIX w. p.n.e.):
\begin{quote}
    Blaba, który jest blaba, nie jest prawdziwym blaba.

    \raggedleft\slshape tłum. z~chińskiego Robert Blarbarucki
\end{quote}

\section{Definitions}

Oto dwie definicje wprowadzające podstawowe pojęcia blabalii
fetorycznej:

\begin{defi}\label{skupienie}
    Silny, zwarty i gotowy fetor bazowy nazwiemy \emph{skupieniem}.
\end{defi}

\begin{defi}\label{fetor}
    \emph{Fetorem} nazwiemy skupienie blaba spełniające następujący
    \emph{aksjomat reperkusatywności}:
    $$\forall \mathcal{X}\in Z(t)\ \exists
        \pi\subseteq\oint_{\Omega^2}\kappa\leftrightarrow 42$$
\end{defi}


\section{Blabalizator różnicowy}

Teoretycy blabalii (zob. np. pracę~\cite{grglo}) zadowalają się
niekonstruktywnym opisem natury fetorów.

Podstawowym narzędziem blabalii empirycznej jest blabalizator
różnicowy.  Przyrząd ten pozwala w~sposób przybliżony uzyskać
współczynniki rozkładu Głombaskiego dla fetorów bazowych
i~harmonicznych.  Praktyczne znaczenie tego procesu jest oczywiste:
korzystając z~reperkusatywności pozwala on przejść do przestrzeni
$\Lambda^{\nabla}$, a~tym samym znaleźć retroizotonalne współczynniki
semi-quasi-celibatu dla klatek Rozkoszy (zob.~\cite{JR}).

Klasyczne algorytmy dla blabalizatora różnicowego wykorzystują:
\begin{enumerate}
    \item dualizm falowo-korpuskularny, a w szczególności
          \begin{enumerate}
              \item korpuskularną naturę fetorów,
              \item falową naturę blaba,
              \item falowo-korpuskularną naturę gryzmołów;
          \end{enumerate}
    \item postępującą gryzmolizację poszczególnych dziedzin nauki, w
          szczególności badań systemowych i rozcieńczonych;
    \item dynamizm fazowy stetryczenia parajonizacyjnego;
    \item wreszcie tradycyjne opozycje:
          \begin{itemize}
              \item duch --- bakteria,
              \item mieć --- chcieć,
              \item myśl --- owsianka,
              \item parafina --- durszlak\footnote{Więcej o tym przypadku --- patrz
                        prace Gryzybór-Głombaskiego i innych teoretyków nurtu
                        teoretyczno-praktycznego badań w~Instytucie Podstawowych
                        Problemów Blabalii w~Fifie.},
              \item logos --- termos%\footnote{Szpotański}
          \end{itemize}
          z właściwym im przedziwym dynamizmem.
\end{enumerate}

\begin{figure}[tp]
    \centering
    \framebox{\vbox to 4cm{\vfil\hbox to
                7cm{\hfil\tiny.\hfil}\vfil}}
    \caption{Artystyczna wizja blaba w~obrazie węgierskiego artysty
        Josipa~A. Rozkoszy pt.~,,Blaba''}
\end{figure}
